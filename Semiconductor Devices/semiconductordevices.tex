

% Summary semiconductor devices D-ITET
% ===========================================================================
% @Author: Noah Huetter
% @Date:   2019-02-20 17:26:28
% @Last Modified by:   noah
% @Last Modified time: 2019-03-13 15:26:54
% ---------------------------------------------------------------------------

\documentclass[a4paper, fontsize=8pt, landscape, DIV=1]{scrartcl}
\usepackage{hyperref}
\usepackage{lastpage}
\usepackage{bm}
% Include general settings and customized commands
%
% General packages and settings
% ===========================================================================
% Author:			Silvano Cortesi (cortesis@student.ethz.ch)
% Version:			1.2
% Last changed:		03.01.2018
%
% ---------------------------------------------------------------------------




\usepackage[german]{babel} %choose your language \usepackage[german]{babel}
%\usepackage[T1]{fontenc}
\usepackage[utf8]{inputenc}
\usepackage{fancyhdr}
%\usepackage{lastpage}
%\usepackage{lmodern}
\usepackage{enumerate}
%\usepackage{float} % for positioning of figures
\usepackage[landscape, margin=1cm]{geometry}
\usepackage[dvipsnames]{xcolor}
\usepackage{pdfpages}


%% Math %%
\usepackage{todonotes}
\usepackage{amscd}
\usepackage{blindtext}
\usepackage{enumitem}
\usepackage{multicol}
\usepackage{parskip}
\usepackage{empheq}
\usepackage{amsmath}
\usepackage{amsfonts}
\usepackage{amssymb}
\usepackage{amsthm}
%\usepackage{dsfont}
%\usepackage{esint} % provides \oiint
\usepackage{mathrsfs}
%\usepackage{trfsigns}
%\numberwithin{equation}{subsection}
%\usepackage{numprint}

%% Graphics & Charts %%
\usepackage{graphicx}
%\usepackage{pdfpages}
%\usepackage{booktabs}
\usepackage{array}
%\usepackage{paralist}
%\usepackage{framed}
%\usepackage{trfsigns}
\usepackage{tikz}
%\usepackage[lofdepth,lotdepth]{subfig}
%\usepackage{tikz}  %Graphen zeichnen
%\usetikzlibrary{decorations.pathmorphing}
%\usetikzlibrary{arrows.meta,arrows}
%\usepackage{pgfplots}
%% General Settings %%
%\setlength{\parindent}{0px}
%\setkomafont{captionlabel}{\normalfont\bfseries}

%\pagestyle{fancy}
%\lfoot{\tiny \today}
%\rfoot{\thepage\  / \pageref{LastPage}}
%\cfoot{}
%\renewcommand{\footrulewidth}{0.4pt}

%% provides command \uline{} for underlining words
%\usepackage{ulem}

%% colour headings
%\usepackage{color}
%\definecolor{bluen}{cmyk}{1,0.5,0,0}
%\definecolor{bloodorange}{cmyk}{0,.92,1,.2}
%\addtokomafont{section}{\color{bloodorange}}
%\addtokomafont{subsection}{\color{bloodorange}}
%\addtokomafont{subsubsection}{\color{bloodorange}}
%\addtokomafont{paragraph}{\small\color{bloodorange}}
%\addtokomafont{subparagraph}{\small\color{bloodorange}}

%% Signs & Special Formating %%
%\usepackage{ulem} %normalem: \emph{Text} is italic again.
%\usepackage{multicol,multirow}
%\usepackage{tabularx}
%\usepackage{stackrel}
%\usepackage{makeidx}
%\usepackage{mparhack} % bessere margiale bei seitenumbruch

% make document compact
\usepackage[compact]{titlesec}
\titlespacing{\section}{0pt}{*0}{*0}
\titlespacing{\subsection}{0pt}{*0}{*0}
\titlespacing{\subsubsection}{0pt}{*0}{*0}

\parindent 0pt
\pagestyle{empty}
\setlength{\unitlength}{1cm}
\setlist{leftmargin = *}

%include also newer PDF
%\pdfminorversion=6

%itemize bullets
\renewcommand\labelitemi{--}

% Set the color of your style
% Avaiable are: Apricot, Aquamarine, Bittersweet, Black, Blue, blue, BlueGreen, BlueViolet, BrickRed, Brown, BurntOrange, CadetBlue, CarnationPink, Cerulean, CornflowerBlue, Cyan, Dandelion, DarkOrchid, Emerald, ForestGreen, Fuchsia, Goldenrod, Gray, Green, GreenYellow, JungleGreen, Lavender, ... (more at: http://en.wikibooks.org/wiki/LaTeX/Colors)
\def\StyleColor{MidnightBlue}

%
% General commands
% ===========================================================================
% Author:			Silvano Cortesi (cortesis@student.ethz.ch)
% Version:			1.2
% Last changed:		03.01.2018
%
% ---------------------------------------------------------------------------

%..ROEMISCHE_ZAHLEN
	\newcommand{\Roe}[1]{\uppercase\expandafter{\romannumeral #1 }}

%..ZAHLENMENGEN
	\newcommand{\N}{\mathbb{N}}
	\newcommand{\Z}{\mathbb{Z}}
	\newcommand{\Q}{\mathbb{Q}}
	\newcommand{\R}{\mathbb{R}}
	\newcommand{\real}{\R}
	\newcommand{\C}{\mathbb{C}}
	\newcommand{\complex}{\C}
	\newcommand{\0}{\mathbb{O}}
	\newcommand{\F}{\mathbb{F}}
	\newcommand{\K}{\mathbb{K}}
    \newcommand{\angstrom}{\textup{\AA}}
    
%..PFEILE
	\renewcommand{\leadsto}{\Longrightarrow}
	\newcommand{\leftrightleadsto}{\Longleftrightarrow}

%..VEKTOREN
	\newcommand{\Ul} {\underline}
	\newcommand{\vEx} {\vec{e}_x}
	\newcommand{\vEy} {\vec{e}_y}
	\newcommand{\vEz} {\vec{e}_z}
	\newcommand{\vEq} {\vec{e_1}}
	\newcommand{\vEw} {\vec{e_2}}
	\newcommand{\vEe} {\vec{e_3}}
	\newcommand{\transpose} {^{\text{T}}}
	\newcommand{\vect}[1]{\boldsymbol{#1}}
	
%..MATRIX
    \newcommand{\MATR}[1]{ \displaystyle \left( \begin{matrix} #1 \end{matrix} \right)}
    \newcommand{\MATRABS}[1]{ \displaystyle \left| \begin{matrix} #1 \end{matrix} \right|}

%..GRAPHICS
  \newcommand{\cgraphic}[2]{\begin{center}\includegraphics[width=#1\columnwidth,keepaspectratio]{#2}\end{center}}
  \newcommand{\cgraphicc}[3]{\begin{center}\includegraphics[width=#1\columnwidth,keepaspectratio]{#2}\captionof{figure}{#3}\end{center}}
  \newcommand{\cgraphicd}[3]{\begin{center}\includegraphics[width=#1\columnwidth,keepaspectratio]{#2}\includegraphics[width=#1\columnwidth,keepaspectratio]{#3}\end{center}}
  
%..FONTS AND LETTERS
  \newcommand*{\rom}[1]{\uppercase\expandafter{\romannumeral #1\relax}}
  \newcommand{\ts}{\textsuperscript}

%..KOMPLEXE ZAHLEN
	\renewcommand{\Re}{\text{Re}\,}
	\renewcommand{\Im}{\text{Im}\,}

%..OPERATOREN
	\DeclareMathOperator{\grad}{grad}
	\renewcommand{\div}{\text{div}\,}
    	\DeclareMathOperator{\rot}{rot}
    	\DeclareMathOperator{\divg}{div}
    	\DeclareMathOperator{\Tr}{Tr}
    	\DeclareMathOperator{\const}{const}
	\DeclareMathOperator{\imag}{i}
	\newcommand{\Lapl}{\hbox{\footnotesize{$\Delta$}}}

%..DIFFERENTIALRECHNUNG
	\newcommand{\Dx} {\,\mathrm{d}}
	\newcommand{\abl}[1] {\frac{\mathrm{d}}{\mathrm{d}#1}}
	\newcommand{\Abl}[2] {\frac{\mathrm{d}#1}{\mathrm{d}#2}}
	\newcommand{\ablq}[1] {\frac{\mathrm{d^2}}{\mathrm{d}#1^2}}
	\newcommand{\Ablq}[2] {\frac{\mathrm{d^2}#1}{\mathrm{d}#2^2}}
	\newcommand{\pabl}[1] {\frac{\partial}{\partial#1}}
	\newcommand{\pablq}[1] {\frac{\partial^2}{\partial#1^2}}
	\newcommand{\Pabl}[2] {\frac{\partial#1}{\partial#2}}
	\newcommand{\Pablq}[2] {\frac{\partial^2#1}{\partial#2^2}}

%..INTEGRALRECHNUNG
	\newcommand{\dint}{\displaystyle{\int}}
	\newcommand{\intab}{\int^b_a}
	\newcommand{\intinf}{\int_{-\infty}^\infty}
	\newcommand{\dintab}{\displaystyle{\int^b_a}}
	\newcommand{\dintpi}{\displaystyle{\int^{\pi}_{-\pi}}}
	\newcommand{\dintzpi}{\displaystyle{\int^{2\pi}_{\mbox{-}2\pi}}}
	\newcommand{\dA}{\hspace{4pt}\mathrm{d}A}
	\newcommand{\dx}{\hspace{4pt}\mathrm{d}x}
	\newcommand{\dy}{\hspace{4pt}\mathrm{d}y}
	\newcommand{\dz}{\hspace{4pt}\mathrm{d}z}
	\newcommand{\dr}{\hspace{4pt}\mathrm{d}r}
	\newcommand{\ds}{\hspace{4pt}\mathrm{d}s}
	\newcommand{\dS}{\hspace{4pt}\mathrm{d}S}
	\newcommand{\dt}{\hspace{4pt}\mathrm{d}t}
	\newcommand{\dm}{\hspace{4pt}\mathrm{d}m}
	\newcommand{\dk}{\hspace{4pt}\mathrm{d}k}
	\newcommand{\dl}{\hspace{4pt}\mathrm{d}l}
	\newcommand{\du}{\hspace{4pt}\mathrm{d}u}
	\newcommand{\dv}{\hspace{4pt}\mathrm{d}v}
	\newcommand{\dV}{\hspace{4pt}\mathrm{d}V}
	\newcommand{\dphi}{\hspace{4pt}\mathrm{d}\varphi}
	\newcommand{\domega}{\hspace{4pt}\mathrm{d}\omega}
	\newcommand{\dvarsigma}{\hspace{4pt}\mathrm{d}\varsigma}
	\newcommand{\dtau}{\hspace{4pt}\mathrm{d}\tau}
	\newcommand{\dtheta}{\hspace{4pt}\mathrm{d}\vartheta}
	\newcommand{\dmu}{\hspace{4pt}\mathrm{d}\mu}
	\newcommand{\dxi}{\hspace{4pt}\mathrm{d}\xi}
	\newcommand{\deta}{\hspace{4pt}\mathrm{d}\eta}
	\newcommand{\dvecl}{\hspace{4pt}\mathrm{d}\vec{l}}
	\newcommand{\dvecS}{\hspace{4pt}\mathrm{d}\vec{S}}

%..LIMES
    \DeclareMathOperator{\limni}{\lim\limits_{n\to\infty}}
    \DeclareMathOperator{\limxi}{\lim\limits_{x\to\infty}}
    \DeclareMathOperator{\limho}{\lim\limits_{h\to0}}
    \newcommand{\limxai}[1]{\ensuremath{\lim\limits_{x\to #1}}}

%..SUMMEN
    \DeclareMathOperator{\sumni}{\sum_{n=0}^{\infty}}
    \newcommand{\sumnia}[1]{\ensuremath{\sum_{n=#1}^{\infty}}}


%..PARTIELLE ABLEITUNG
    \DeclareMathOperator{\partf}{\dfrac{\partial f}{\partial x}}
    \newcommand{\partfo}[1]{\ensuremath{\dfrac{\partial f}{\partial #1}}}
    \newcommand{\parto}[1]{\ensuremath{\dfrac{\partial }{\partial #1}}}
    \newcommand{\partt}[2]{\ensuremath{\dfrac{\partial^2 }{\partial #1\partial #2}}}
    \newcommand{\partq}[1]{\ensuremath{\dfrac{\partial^2 }{\partial #1^2}}}


%..ENUMERATION
    \newenvironment{abc}{\begin{enumerate}[(a)]}{\end{enumerate}}
    \newenvironment{cabc}{\begin{compactenum}[(a)]}{\end{compactenum}}
    \newenvironment{romanenum}{\begin{enumerate}[i.]}{\end{enumerate}}
    \newenvironment{cromanenum}{\begin{compactenum}[i.]}{\end{compactenum}}

%..FUNCTIONS
    \DeclareMathOperator{\arsinh}{arsinh}
    \DeclareMathOperator{\arcosh}{arcosh}
    \DeclareMathOperator{\artanh}{artanh}
    \DeclareMathOperator{\arcoth}{arcoth}
    \DeclareMathOperator{\arccot}{arccot}
    \DeclareMathOperator{\Arg}{Arg}
    \DeclareMathOperator{\Log}{Log}
    \newcommand{\dis}[1]{\hspace{#1cm}}
    \newcommand{\abs}[1]{\ensuremath{\left\vert#1\right\vert}}
    \newcommand{\attention}{\raisebox{-1pt}{{\makebox[1.6em][c]{\makebox[0pt][c]{\raisebox{.13em}{\small!}}\makebox[0pt][c]{\color{red}\Large$\bigtriangleup$}}}}}
    \DeclareMathOperator{\meq}{\stackrel{!}{=}}
    
    
% section color box
\setkomafont{section}{\mysection}
\newcommand{\mysection}[1]{%
    \Large\sffamily\bfseries%
    \setlength{\fboxsep}{0cm}%already boxed
    \colorbox{\StyleColor!40}{%
        \begin{minipage}{\linewidth}%
            \vspace*{2pt}%Space before
            #1
            \vspace*{-1pt}%Space after
        \end{minipage}%
    }}

%subsection color box
\setkomafont{subsection}{\mysubsection}
\newcommand{\mysubsection}[1]{%
    \normalsize \sffamily\bfseries%
    \setlength{\fboxsep}{0cm}%already boxed
    \colorbox{\StyleColor!20}{%
        \begin{minipage}{\linewidth}%
            \vspace*{2pt}%Space before
             #1
            \vspace*{-1pt}%Space after
        \end{minipage}%
    }}

% highlighter
\newcommand{\hilight}[1]{\colorbox{\StyleColor}{#1}}
\newcommand{\highlighty}[1]{%
  \setlength{\fboxsep}{0pt}\colorbox{yellow!100}{\ensuremath{#1}}}

\newcommand{\highlightg}[1]{%
  \setlength{\fboxsep}{0pt}\colorbox{green!100}{\ensuremath{#1}}}

\newcommand{\highlightbg}[1]{%
   \colorbox{green!100}{$\displaystyle #1$}}  

% equation box        
\newcommand{\eqbox}[1]{\setlength{\fboxrule}{1mm}\fcolorbox{\StyleColor}{white}{\hspace{0.5em}$\displaystyle#1$\hspace{0.5em}}}

%center equationbox
\newcommand{\ceqbox}[1]{\vspace*{4pt} \begin{center}\eqbox{#1}\end{center}\vspace*{4pt}}

% Tabular
\newcolumntype{P}[1]{>{\centering\arraybackslash}p{#1}}


% \bibliography{semiconductordevices}
% \bibliographystyle{ieeetr}

%change page style for header
\pagestyle{fancy}
\footskip 20pt

 
% -----------------------------------------------------------------------
\IfFileExists{../build/revision.tex}{
  \input{../build/revision.tex}
  \rhead{Compiled: \compiledate \hspace{1em} on \hostname \hspace{1em} from commit: \revision \hspace{1em} Noah Huetter}
}{\rhead{Noah Huetter, Daniel Raimundo}}

\lhead{ETH Semiconductor Devices 2019}
\chead{\thepage}
\cfoot{}
\headheight 17pt \headsep 10pt
\title{ETH Semiconductor Devices 2019}
\author{Noah Huetter, Daniel Raimundo}

\date{\today}
\begin{document}

\setcounter{secnumdepth}{2} %no enumeration of sections
\begin{multicols*}{4}
	\section*{Disclaimer}
	This summary is part of the lecture ``ETH Semiconductor Devices'' by Prof. Dr. Colombo Bolognesi (FS19). It is based on the lecture. \\[6pt]
	Please report errors to \href{mailto:huettern@student.ethz.ch}{huettern@student.ethz.ch} such that others can benefit as well.\\[6pt]	
  The upstream repository can be found at \href{https://github.com/noah95/formulasheets}{https://github.com/noah95/formulasheets}
	\vfill\null
	\pagebreak
%  \maketitle 
  \thispagestyle{fancy}

  % ---------------------------------------------------------------------------
  \section{Introduction}
  % ---------------------------------------------------------------------------
  \setcounter{page}{1}
    \subsection{Electric resistivity/conductivity}
    \cgraphic{1}{img/conductor.png}
    \begin{itemize}[noitemsep,nolistsep]
       \item$\text{Conductivity}: \sigma \left[S/m\right]$;
       \item$\text{Resistivity}: \rho=\sigma^{-1} \left[\Omega\cdot m\right]$;
       \item$\text{Resistance}: R=\rho\frac{l}{A}=\rho\frac{l}{r^2\pi} \left[\Omega\right]$.
	\end{itemize}
    \cgraphic{1}{img/resistivity.png}

    \subsection{Electron motion}
    \cgraphic{0.5}{img/electronmotion.png}
    \begin{itemize}[noitemsep,nolistsep]
      \item Electric Force: $\vec{F}_e = Q\vec{E} \quad [\vec{E}]=\frac{\text{N}}{C}=\frac{V}{m}$;
      \item Magnetic Force:
        $\vec{F}_m = Q\vec{v}\times\vec{B}$;
      \item Electric field: $F_{12}=F_{21}=\frac{1}{4\pi\epsilon_0}\frac{Q_1Q_2}{r^2}$;
      \item Current: $I = \abl{t}Q \quad [Q]=\frac{C}{s}={s\cdot A}$.
    \end{itemize}

    \subsection{Current flow}
    \cgraphic{0.5}{img/current.png}
    
    \subsection{Moore's Law}
    ML: Number of transistors on an IC doubles every two years$\rightarrow p(t) = p_0 \cdot b^{t/\tau}$
	\begin{itemize}[noitemsep,nolistsep]
		\item$p(t) = \text{population at given time}$;
		\item$p_0 = \text{initial population}$;
		\item$b = \text{growth rate per time constant}$;
		\item$\tau = \text{time constant}$.
	\end{itemize}
      

  % ---------------------------------------------------------------------------
  \section{Solid state physics}
  % ---------------------------------------------------------------------------
  \subsection{Crystal structures}
  \begin{itemize}[noitemsep,nolistsep]
  \item\textbf{CN (Coordination Number)}: Number of nearest neighbours any atom has in a given crystal lattice, which is periodic in 3D. $a$: lattice cst $[m]$;
  \item\textbf{$\bm{N_{eff}}$ (Effective N of atoms in cell)}$=N_{complete}+N_{face}/2+N_{edge}/8[atoms]$;
  \item\textbf{VD (Volume Density)}= $\frac{N_\text{atoms\_in\_cell}}{V_\text{unit\_cell}}\left[\frac{at}{m^3}\right]$;
  \item\textbf{VPD (V Packing D)}= $\frac{V_\text{atoms\_in\_cell}}{V_\text{unit\_cell}}[1]$;
  \item\textbf{SD (Surface Density)}= $\frac{N_\text{atoms\_on\_plane}}{S_\text{plane}}\left[\frac{at}{m^2}\right]$;
  \item\textbf{SPD (S Packing D)}= $\frac{S_\text{atoms\_on\_plane}}{S_\text{plane}}[1]$;
  \item\textbf{Types of 3D periodic stuctures:}\textbf{sc (Simple Cubic)}: Po; \textbf{bcc (Body Centered Cubic)}: Li, Na, K, Cr, Fe, Nb, Mo; \textbf{fcc (Face Centered Cubic)}: Al, Ar, Ni, Cu, Kr, Pd, Ag, Xe, Ta, W, Pt, Au; \textbf{dia (Diamond)}: Si.
	\end{itemize}
  \cgraphic{0.5}{img/wholecrystal.png}
  \begin{center}
    \begin{tabular}[h]{c c c c c c}
    Fig & Type & $N_{eff}$ & CN & $r_a$ & VPD\\\hline	
	(a) & sc & 1 & 6 & $a/2$ & $\pi/6$\\
	(b) & bcc & 2 & 8 & $\sqrt{3}a/4$ & $\sqrt{3}\pi/8$\\
	(c) & fcc & 4 & 12 & $\sqrt{2}a/4$ & $\sqrt{2}\pi/6$\\
	$\downarrow$ & dia & 8 & 4 & $\sqrt{3}a/8$ & $\sqrt{3}\pi/16$\\
  \end{tabular}
	\end{center}
  \cgraphic{1}{img/crystalstructures.png}


  \subsection{Silicon}
  \cgraphic{1}{img/siliconstructure.png}
  \textbf{Diamond unit cell}: (l) cubic unit cell; (r) inherent tetrahedral structure. For Si, the nuclear diameter is $7.2\text{fm} = 2.7\cdot 10^{-6}\text{nm}$: Matter is impressively ``empty''. \textbf{Tetrahedral bonding angle} $=109.471$ deg.

  \subsection{Crystal Planes and Directions}
  \cgraphic{0.8}{img/planes.png}
  Crystals have different periodicities in different directions.
  \textbf{Miller indices} $[abc] = \left[\frac{1}{p},\frac{1}{q},\frac{1}{r}\right]$, $p,q,r$: intersections with the $x,y,z$ axis.

  \subsection{Elements}
  \cgraphic{1}{img/elements.png}
  \textbf{Electronegativity}: Tendency to attract electrons. Increases from Bottom to Top / Left to Right.

  % \subsubsection{Simple Metals}
  % (\textit{e.g.} Alkali Metals, like Na) Collective interaction of mobile electron fluid with positive metal ions. 
  % \textbf{Occurs when teh coordination number is greater than the nu,ber of valenve electrons}. Close packed structures. The material is mechanically soft. 
  % \subsubsection{Transition Metals}
  % (\textit{e.g.} Mo, W) Bon is covalent-like. Transision metals are much harder than simple metals and located in the center of the transition metals row.
  % \subsubsection{Transition Metals}
  % (Column \rom{4}, \rom{3}-\rom{5}, \rom{2}-\rom{6}) Based on hybridization of ``s'' and ``p'' orbitals. 
  % \textbf{Bonds are very directional}, material is mechanically stiff (brittle). Generally habe a diamond-like crystal structure.

  \subsection{Band Gap}
  As atoms approach eachother, atomic energy states spread from sharp levels into bands, which are separated by \textbf{energy gaps} where no $e^-$ can exist.
  \cgraphic{.85}{img/bandgap.png}
  In solids, the atoms' outermost (\textit{i.e.} valence) $e^-$ determine bonding and electronic properties.

\begin{itemize}[noitemsep,nolistsep]
  \item\textbf{Partially Filled/Empty Band}:\textit{conduction}. As atoms approach each other to form a solid, valence $e^-$ distributions overlap. Equilibrium distance ~ maximum density of electrons for isolated atoms. Lowering of potential barriers between atoms allows electrons to move freely.

  \item\textbf{Completely Full/Empty Band} \textit{insulation}. If full, $e^-$ are there but no net current. At equilibrium atomic separation, bands are seperated by a forbidden energy gap. At low T, valence band full, conduction band empty: \textit{i.e.} no current.
\end{itemize}
\subsection{Effective mass}
Mass that a particle seems to have when reponding to forces. Generally in units of the rest mass of an $e^-$/hole.
\ceqbox{\frac{1}{m^*_{n,p}}=\frac{1}{\hbar^2}\parto{k^2}E_{C,V}\quad [m_{e,h}^{-1}]}
\subsection{Density of states (DOS)}
$D(E)=g(E):\quad \frac{N_\text{available\_states}}{\text{Volume}\cdot\text{Energy}}$. $D(E)dE$ density of states in range $[E,E+dE]$, in $[m^{-3}]$.
\ceqbox{D_{C,V}(E)=\frac{8\pi\sqrt{2}}{h^3}\left(m^*_{n,p}\right)^{3/2}\sqrt{\abs{E-E_{C,V}}}}

\subsection{Fermi-Dirac function (FD) / MB approx.}
\cgraphic{1}{img/fermidirac.png}
Probability that an an allowed  state is occupied by an $e^-$ (by a hole: $1-f(E)$): 
\ceqbox{f(E)=\frac{1}{1+e^{\frac{E-E_F}{kT}}}\quad \{E_F\in\R:f(E)=0.5\}}
The Maxwell-Boltzmann approximation (MB) can be used if $E-E_F>3kT$ (approx. error $<5\%$):
\ceqbox{f(E)\stackrel{MB}{\cong} e^{-\frac{E-E_F}{kT}}}

\subsection{$e^-$/hole density, effective DOS}\label{n0p0}
$[n_0]=[p_0]=[N_{C,V}]=m^{-3}$
\ceqbox{n_0=\int_{E_C}^{\infty}f(E)D_C(E)dE\stackrel{MB}{\cong} N_C e^{-\frac{E_C-E_F}{kT}}}
\ceqbox{p_0=\int_{-\infty}^{E_V}(1-f(E))D_V(E)dE\stackrel{MB}{\cong} N_V e^{-\frac{E_F-E_V}{kT}}}
\ceqbox{N_{C,V}=\frac{4\sqrt{2}\left(\pi m^*_{n,p}kT\right)^{3/2}}{h^3}}
\subsection{Intrinsic carriers}
\cgraphic{1}{img/intrinsiccarriers}
Two types of carriers: $e^-$ and holes. Intrinsic carriers are the thermally generated $e^-$-hole pairs ($n_0=p_0=n_i$) in a pure(undoped) crystal, related to the smearing of the FD for increasing $T$. $\rightarrow$ two partially filled bands, both the C and V bands can now carry current.

\textbf{Mass Action Law} from eqs. in Sec.\ref{n0p0} ($E_C-E_V=E_g$):
\ceqbox{n_i^2 = n_0 p_0 \stackrel{MB}{\cong} N_C N_V e^{-\frac{E_C-E_V}{kT}}}
\textbf{Dependence on temperature}: $N_C,N_V \propto T^{3/2}$ (negl.: $E_g$ also depends from $T$):
\cgraphic{0.8}{img/intrinsiccarrierdensity}


  % ---------------------------------------------------------------------------
  \section{Doping}
  % ---------------------------------------------------------------------------
  Doping is the introduction of impurities with different atomic valence in the pure crystal. These impurities are called \textit{Extrinsic Carriers}.


%  \begin{tabular}[h]{l l}
%    $E_D$   & Donor energy level/state \\
%    $E_c$   & Conduction band edge\\
%    $E_v$   & Valence band edge\\
%    $E_F$   & Fermi energy\\
%    $E_{Fi}$   & Intrinsic Fermi energy\\
%    $E_g$   & Band gap width/energy\\
%    $m^*$   & Effective mass of electron(n)/hole(p) \\
%    $\epsilon_0$ & Vacuum permittivity $8.854\cdot 10^{-12}$F/m\\
%    $n_i$   & Intrinsic electron concentration \\
%    $p_i$   & Intrinsic hole concentration\\
%    {}      & $n_i=p_i$\\
%    $n_0$   & Thermal-equilibrium $e^-$ concentration\\
%    $p_0$   & Thermal-equilibrium hole concentration\\
%    $n_d$   & Concenctraiton of $e^-$ in donor state \\
%    $p_a$   & Concenctraiton of holes in acceptor state \\
%    $N_d$   & Concentration of donor atoms\\
%    $N_a$   & Concentration of acceptor atoms\\
%    $N_c$   & Effective density of states\\
%    $N_v$   & Effective density of states\\
%    $N_d^+$   & Conc. of pos. charged (ionized) donors \\
%    $N_a^-$   & Conc. of neg. charged (ionized) acceptors \\
%    % $$   &  \\
%  \end{tabular}

  \subsection{Impurity doping}
  It is distinguished between negative and positive doping.
  
  \textbf{N-Doping} $n_0>p_0$
  \cgraphic{0.7}{img/ndope}
  Taking donors from PT col \rom{5} (P, As, Sb) that have 5 valence electrons. 4 make covalent bond, 5\ts{th} is ``not needed'' and looks like a hydrogen atom. It is easily inozed/excited to conduction band. Its energy state is close to CB ($E_C$).
  \attention{Overall, the solid is still charge neutral but with impurities}\attention
  
  Donors introduce an energy state $E_D$ near the conduction band edge $E_C$. $e^-$ easily promoted to conduction band because $E_C-E_D<E_C-E_V=E_g$ \textbf{Extra $e^-$ are added without adding holes.}

  \textbf{P-Doping} $p_0>n_0$
  \cgraphic{0.7}{img/pdope}
  Same as n-doping but by adding extra holes. Acceptors from PT col \rom{3} (B, Al, Ga, In)

  \textbf{Energy bands}
  \cgraphic{1}{img/dopeband}
    \begin{tabular}[h]{p{0.43\linewidth} | p{0.55\linewidth}}
    \textbf{n-type doping} & \textbf{p-type doping} \\
    Donor state $E_D<E_C$ & Acceptor state $E_A>E_V$ \\

  \end{tabular}

  \cgraphic{1}{img/bandenergies}
  Impurity levels in Silicon in eV: amount of energy is required to move impurity to conductance/valence band.
  
  \textbf{Degenerate n/p Semiconductor}: $E_F$ is over $E_C$/under $E_V$.
\subsection{Intrinsic Fermi Level}
$E_i\equiv E_F$ in an undoped(=intrinsic) semiconductor.
\ceqbox{E_{i}=\frac{E_V+E_C}{2}+\frac{kT}{2}ln\left(\frac{N_V}{N_C}\right)}

\ceqbox{E_{i} - E_{\text{midgap}} = \frac{3}{4}kT\ln\left(\frac{m_p^*}{m_n^*}\right)}

\ceqbox{n_i=N_{C,V}e^{-\frac{\abs{E_{C,V}-E_i}}{kT}}}

%\ceqbox{E_F-E_{i}=kT ln\left(\frac{\{n,p\}_0}{n_i}\right)}

\ceqbox{\{n,p\}_0=n_i e^{\pm\frac{E_F-E_{i}}{kT}}}

\subsection{Electroneutrality}
Charge balance equation:
\ceqbox{n_0 + N_a^- = p_0 + N_d^+}
Assuming complete ionization of dopants and equilibrium, use for majority carriers:
\ceqbox{\{n,p\}_0 = \frac{\abs{N_d-N_a}}{2}+\sqrt{\left(\frac{N_d-N_a}{2}\right)^2+n_i^2}}
\begin{itemize}[nolistsep,noitemsep]
	\item $N_d>N_a$: n-type compensated semiconductor;
	\item $N_d<N_a$: p-type compensated semiconductor;
	\item $N_a=N_d$: completely compensated semiconductor, with chars. of an intrinsic material.
\end{itemize}
\subsection{Amount of ionized donors/acceptors}
\begin{enumerate}[nolistsep,noitemsep]
	\item Compute $E_F$ assuming complete ionization, and using that the percentage of ionized dopants;
	\item Compute the probability that the donors are ionized:
	\begin{itemize}[nolistsep,noitemsep]
		\item For n-type: Donor ionized $N_D^+\rightarrow$ hole with prob. $1-f(E_D)$;
		\item For p-type: Acceptor ionized $N_A^-\rightarrow$ $e^-$ with prob. $f(E_A)$.
	\end{itemize}
	\item Complete ionization hypothesis justified if more than $95\%$ of dopants ionized.
\end{enumerate}

%  \subsection{Fermi Level \& Doping}
%  
%  \subsection{Book Chapter 4 Summary}
%
%    \textbf{Materials at 300K} \\
%      \begin{tabular}[h]{l c c c}
%        Material & Si & Ge & GaAs \\
%        $n_i^2(cm^{-6})$  & $9.3\cdot 10^{19}$  & $5.76\cdot 10^{26}$ & $3.24\cdot 10^{12}$ \\
%        $N_c(cm^{-3})$    & $2.86\cdot 10^{19}$ & $1.04\cdot 10^{19}$ & $4.7\cdot 10^{17}$ \\
%
%        $N_v(cm^{-3})$    & $1.04\cdot 10^{19}$ & $6.0\cdot 10^{18}$  & $7.0\cdot 10^{18}$ \\
%        $E_g (eV)$        & $1.12$ & $0.66$ & $1.42$ \\
%        $m_n^*/m_0$       & $1.08$ & $0.067$ & $0.55$\\
%        $m_p^*/m_0$       & $0.56$ & $0.48$ & $0.37$\\
%      \end{tabular}
%
%    \textbf{Intrinsic Band Gap} \\
%      \begin{align*}
%        E_g &= -kT\ln\frac{n_i^2}{N_c N_v} &
%        n_i^2 &= N_v N_v e^{-\frac{E_g}{kT}} \\
%        n_0 &= N_ce^{\frac{-(E_c-E_F)}{kT}} &
%        p_0 &= N_ve^{\frac{-(E_F-E_v)}{kT}}
%      \end{align*}
%
%    \textbf{Effective Mass} \\
%      \begin{align*}
%        N_c &= 2\left(\frac{2\pi m_n^*kT}{h^2}\right)^{3/2} &
%        N_v &= 2\left(\frac{2\pi m_p^*kT}{h^2}\right)^{3/2} &
%      \end{align*}
%
%    \textbf{Intrinsic Carrier Concentration} \\
%      \begin{align*}
%        n_i^2 &= N_cN_ve^{\frac{-(E_c-E_v)}{kT}} = N_cN_ve^{\frac{-E_g}{kT}}
%      \end{align*}
%
%    \textbf{Intrinsic Fermi Level} \\
%      \begin{align*}
%        E_{Fi} &= \frac{E_v+E_c}{2} + \frac{kT}{2}\ln\left(\frac{N_v}{N_c}\right)
%      \end{align*}
%
%    \textbf{Equilibrium distribution of el/holes} \\
%      \begin{align*}
%        n_0 &= n_i e^{\frac{E_F-E_{Fi}}{kT}} &
%        p_0 &= n_i e^{\frac{-(E_F-E_{Fi})}{kT}} &
%      \end{align*}
%
%    \textbf{The $n_0p_0$ product} \\
%      \ceqbox{n_0p_0 = n_i^2}
%
%    \textbf{Statistics of Donors and Acceptors} \\
%      \begin{align*}
%        n_d &= \frac{N_d}{1+\frac{1}{2}e^{\frac{E_d-E_F}{kT}}} &
%        p_d &= \frac{N_a}{1+\frac{1}{2}e^{\frac{E_F-E_a}{kT}}} \\
%      \end{align*}
%
%    \textbf{Thermal-Equilibrium El. Concentration} \\
%    $n_0$ for $N_d>N_a$ (n-type), $p_0$ for $N_a>N_d$ (p-type).
%      \begin{align*}
%        n_0 & = \frac{N_d-N_a}{2}+\sqrt{\left(\frac{N_d-N_a}{2}\right)^2+n_i^2} \\
%        p_0 & = \frac{N_a-N_d}{2}+\sqrt{\left(\frac{N_a-N_d}{2}\right)^2+n_i^2} &
%      \end{align*}
% 
%    \textbf{Position of Fermi Energy Level} \\
%    Use the $n_0$ formula for n-type, the $p_0$ formula for p-type.
%      \begin{align*}
%        E_F-E_{Fi} &= kT\ln\left(\frac{n_0}{n_i}\right) &
%        E_{Fi}-E_{F} &= kT\ln\left(\frac{p_0}{n_i}\right) &
%      \end{align*}

  % ---------------------------------------------------------------------------
  \section{Excess Carriers}
  % ---------------------------------------------------------------------------
  % \subsection{Variables}
  \cgraphic{0.9}{img/dopingvars.png}

    \begin{itemize}[nolistsep, noitemsep]
    \item  $G_{th}$: Thermal generation rate;
    \item  $G_{L}$: Photonic generation rate;
    \item  $R_{th}$: Thermal recombination rate;
    \item  $\tau_{p,n}$: Minority carrier lifetime;
    \item  $\Delta {p,n}$: Excess carrier concentration to equil.;
    \item  $N_t$: Density of recombination centers;
    \item  $\sigma$: Recomb. center cross section \textbf{or} conductivity;
    \item  $v_{th}$: Carrier mobility in thermal equilibrium;
    \item  $R_a,R_b$: $e^-$ indirect capture and emission rate;
    \item  $R_c,R_d$: Hole indirect capture and emission rate;
    \item  $e_{p,n}$: $e^-$/Hole indirect emission probability;
    \item  $U$: Net recombination rate;
    \item  $J_{\text{diff}}$: Diffusion current dens.;
    \item  $D$: Diffusion constant;
    \item  $q$: $e^-$ charge;
    \item  $J_n$: $e^-$ diffusion current dens.;
    \item  $v_{dr,n/p}$: $e^-$/hole drift velocity;
    \item  $J_{dr,n/p}$: $e^-$/hole drift current dens.;
    \item  $\mu_{n/p}$: $e^-$/hole mobility;
    \item  $J_{n/p}$: $e^-$/hole total current dens.
    \end{itemize}

\subsection{Definitions}
\begin{itemize}[nolistsep, noitemsep]
	\item G/R occur to restore the semiconductor's equilibrium state;
	\item G: $e^-$-hole pairs created, occurs when there's a deficit of carriers;
	\item R:$e^-$-hole pairs destroyed, occurs when there's an excess of carriers.
\end{itemize}

\textbf{Non-equilibrium $e^-$,hole concentration}:
\ceqbox{\{n,p\}=\{n,p\}_0+\Delta\{n,p\}}

\textbf{Each G/R event requires a pair}:
\ceqbox{\Delta n=\Delta p}

\begin{itemize}[nolistsep, noitemsep]
	\item Equilibrium $\rightarrow \Delta n=\Delta p=0$;
	\item Steady-state $\rightarrow \Delta n=\Delta p=cst$;
	\item Deficit/Excess of carriers $\Delta n=\Delta p\lesseqgtr0$.
\end{itemize}

\subsection{Thermal equilibrium vs. Steady-state}
\textbf{Thermal equilibrium:}
\begin{itemize}[noitemsep,nolistsep]
	\item No external forces/fields/light/voltage/$\grad T$;
	\item No change over time;
	\item Flat Fermi level;
	\item No net current ($J_{tot}=0$);
	\item $n_i^2=n_0p_0$.
\end{itemize}
\textbf{Stedy-state (SS):}
\begin{itemize}[nolistsep,noitemsep]
	\item No change of external forces over time;
	\item $np\neq n_i^2$;
	\item $\rightarrow \Delta_n=\Delta_p=cst$.
\end{itemize}
Systems always try to go to a SS, if external influences are removed, they'll try to go to equilibrium.
\subsection{Direct Generation/Recombination}
\cgraphic{0.8}{img/genrecomb.png}
Thermal (spontaneous) and external generation (e.g. light) accross the energy gap.

\textbf{In equilibrium}: $e^-$ continually generated due to thermal energy. Some electrons recombine with holes, so that on average $n_0$ and $p_0$ are constant. 
\ceqbox{G = R = \beta(n_0p_0) = \beta n_i^2}

\textbf{Out of equilibrium}:

\textit{Recombination rate}:
\ceqbox{R = \beta(np)=\beta(n_0+\Delta n)(p_0+\Delta p)}
\textit{Thermal and Total generation rates}:
\ceqbox{G_{th}=R_{th}=\beta n_0 p_0 \quad G=G_L+G_{th}}
\textit{Net recombination rate}:
\ceqbox{U=R-G_{th}}

\textbf{Mass action law for n and p doping}:
\ceqbox{p_{n0} \ll n_{n0} \quad p_{p0} \gg n_{p0}}
\textbf{Low level injection for n and p doping}:
\ceqbox{\Delta p \ll n_{n0} \quad p_{p0} \gg \Delta n}
(Doping type doesn't change with perturbation)

Last two inequs. are used to approximate the \textit{$e^-$ in p-type/Hole in n-type}:

-\textbf{Minority Carrier Recombination Rates}:
\ceqbox{U_n=\Delta n_p/\tau_n \quad U_p=\Delta p_n/\tau_p}

-\textbf{Direct recombination rate}
\ceqbox{U_{dir}=\beta p_{p0}\Delta n \quad U_{dir}=\beta n_{n0}\Delta p}

-\textbf{Minority carrier lifetime}:
\ceqbox{\tau_n = \left(\beta p_{p0}\right)^{-1} \quad \tau_p = \left(\beta n_{n0}\right)^{-1}}

\subsection{Light on an n-type semiconctor}
\textbf{Light on($t<0$), $U = G_L$}:
\ceqbox{U = \frac{p_n-p_{n0}}{\tau_p} \quad p_n(t < 0) = p_{n0} + \tau_pG_L}
      
\textbf{Light off($t\geq 0$) $G_L=0$}:
$$\Abl{p_n}{t} = G_{th} - R = -U = -\frac{p_n-p_{n0}}{\tau_p}\Rightarrow$$
\ceqbox{p_n(t) = p_{n0} + \tau_p G_L \exp\left(-\frac{t}{\tau_p}\right)}

\subsection{Indirect Generation/Recombination}
\cgraphic{0.7}{img/trap.png}
Energy trap in bandgap:
\ceqbox{U \approx  \frac{v_{th}\sigma_0N_t\cdot\Delta p}{1 + \frac{2n_i}{n_{n0}}\cosh\frac{E_t - E_i}{kT}} = \frac{\Delta p}{\tau_p} = \frac{p_n - p_{n0}}{\tau_p}}

Where $N_tv_{th}\sigma$ are the recombination events taking place per unit time, $v_{th}\sigma$ a cylindrical colume in material per unit time.
\[\frac{1}{2}m_nv_{th}^2=\frac{3}{2}kT \quad v_{th}\approx 10^7 cm\cdot s^{-1}\]
Electron capture ($R_a$) and emission ($R_b$) rate must be equal in therm. equi.
\[R_a = nN_t(1-f(E_t))\cdot v_{th}\sigma_n \quad R_b = e_nN_tf(E_t)\]
The emission probability increases exponentially as $E_t$ gets closer to conduction band edge:
\[e_n = \frac{v_{th}\sigma_nn(1-f(E_t))}{f(E_t)}=v_{th}\sigma_nn_ie^{(E_t-E_i)/kT}\]
For holes:
\[R_c = pN_tf\cdot v_{th}\sigma_p \quad R_d = e_pN_t(1-f)\]
\[e_p = v_{th}\sigma_p n_i e^{(E_i-E_t)/kT}\]
These lead to the equation for $U=R_a-R_b=R_c-R_d$ ($\sigma_p=\sigma_n=\sigma_0$).

\subsection{Direct and Indirect recombination}
\begin{itemize}[nolistsep,noitemsep]
	\item Occur in parallel, depends on which is faster/ has smaller $\tau$;
	\item $U>0$ for execessive minority carriers $rightarrow$ Recombination;
	\item $U>0$ for deficient minority carriers $rightarrow$ Generation.
\end{itemize}
\section{Carrier transport mechanisms}
\subsection{Diffusion}
Result of concentration gradients. Equal probability of moving in any direction $\rightarrow$ flow from high to low concentrations. Fick's first law of diffusion:
\ceqbox{j_{\text{diff}} = -D\Delta N = -D \left( \Pabl{N}{x}\vec{x_u} + \Pabl{N}{y}\vec{y_u} \dots \right)}
In thermal equi. and uniform distribution, free charge carriers are in constant motion; net current is thus zero. Statistical mechanics show that particles at temp. $T$ have avg. thermal energy of $3kT/2$. For a particle of mass $m$ this corresponds to an avg. thermal velocity of:
$\frac{1}{2}m*v_{th}^2=\frac{3}{2}kT$

\textbf{Carrier diffusion currents}:
\ceqbox{j_{\text{diff}\{n,p\}} = -qF = \pm qD_{n,p}\Abl{n}{x}[A/m^2]}

\textbf{Diffusivity=f(Mobility)}:
\ceqbox{D_x\stackrel{MB}{\cong}\frac{kT}{q}\mu_x[m^2/s]}


\subsection{Drift}
Result of an electric field as driving force. Zero field: Electrons move thermally (randomly) in all directions (no net flow) Non-zero field: There is a net drift of electrons, opposite to the E-field. We can then define a drift velocity of electrons $v_{dr,n}$. Electrons do not accelerate indefinitely due to collisions.

\textbf{Drift current of carriers in low E-Fields}:
\ceqbox{j_{dr,\{n,p\}} = \mp qnv_{dr,\{n,p\}}\quad v_{dr,\{n,p\}} = \mp \mu_{\{n,p\}}E}

\textbf{Total drift current}:
\ceqbox{j_{dr,tot}= \sigma E\quad \sigma = q(n\mu_n + p\mu_p)}

\subsection{Total current}
\textbf{Total current = drift + diffusion current = electron + hole current}

\textbf{Electrons:} \ceqbox{j_n = nq\mu\vec{E}+qD_n\Abl{n}{x}}
\textbf{Holes:} \ceqbox{j_p = pq\mu\vec{E}-qD_p\Abl{p}{x}}
\textbf{Total:} \ceqbox{j = j_n + j_p}

  % ---------------------------------------------------------------------------
  \section{Excess Carriers}
  % ---------------------------------------------------------------------------
  % \subsection{Variables}
    \begin{tabular}[h]{l l}
      $G_{n/p}$   & Generation rate of el/hole \\
      $R_{n/p}$   & Recombination rate of el/hole \\
      $J_{n/p}$   & Current density of el/hole \\
      $L_{n/p}$   & Minority carrier diffusion length \\
      $D_{n/p}$   & Diffusion constant \\
      $\mu_{n/p}$ & Carrier mobility \\
    \end{tabular}

  \subsection{Continuity equation}
\textbf{Objective:} Accounting for carrier densities when drift, diffusion and G/R take place. 
\cgraphic{0.5}{img/continuity.png}

The conservation of carriers results in:

\textbf{Change in electrons:}
\ceqbox{\Pabl{n}{t} = \frac{1}{q}\Pabl{j_n}{x}+(G_m-R_n)}

\textbf{Change in holes:}
\ceqbox{\Pabl{p}{t} = -\frac{1}{q}\Pabl{j_p}{x}+(G_p-R_p)}

Inserting the total current equations:

\textbf{Electrons in p-type material}:
\begin{align*}
\Pabl{n_p}{t} =& n_p\mu_n\Pabl{{E}}{x}+\mu_n{E}\Pabl{n_p}{x} + D_n\Pablq{n_p}{x} \\ &+ G_n - \frac{n_p-n_{p0}}{\tau_n}
\end{align*}

\textbf{Holes in n-type material}:
\begin{align*}
\Pabl{p_n}{t} =& p_n\mu_p\Pabl{{E}}{x}+\mu_p{E}\Pabl{p_n}{x} + D_p\Pablq{p_n}{x} \\ &+ G_p - \frac{p_n-p_{n0}}{\tau_p}
\end{align*}

\textbf{Steady state: $\pabl{n_p}{t}=\pabl{p_n}{t}=0$}
\ceqbox{0=D_n\Pablq{n_p}{x}+G_n-\frac{n_p-n_{p0}}{\tau_n}}
\ceqbox{0=D_p\Pablq{p_n}{x}+G_p-\frac{p_n-p_{n0}}{\tau_p}}

\textbf{SS, no E-Field and no ext. carrier G}:
\ceqbox{0=D_n\Pablq{n_p}{x}-\frac{n_p-n_{p0}}{\tau_n}}
\ceqbox{0=D_p\Pablq{p_n}{x}-\frac{p_n-p_{n0}}{\tau_p}}

\begin{center}
\attention \textbf{Steady State $\neq$ Equilbrium} \attention
\end{center}

\subsection{Steady State Surface Generation}
n-type semiconductor with all generation at $x=0$.
\ceqbox{L_p = \sqrt{D_p\tau_p}}

\textbf{General case}
\cgraphicd{0.49}{img/finitesample1.png}{img/finitesample2.png}
\ceqbox{p_n(x) = p_{n0} + (p_n(0)-p_{n0}) \frac{\sinh\left(\frac{W-x}{L_p}\right)}{\sinh\left(\frac{W}{L_p}\right)}}

\textbf{Semi-infinite sample: $L_P \ll W$}
\cgraphicd{0.49}{img/semiinf1.png}{img/semiinf2.png}
\ceqbox{p(x) = p_{n0} + (p_n(0)-p_{n0}) \exp\left(-\frac{x}{L_p}\right)}
    
\textbf{Finite sample:  $L_P \gg W$}
\ceqbox{p_n(x) = (p_n(0)-P_{n0})\frac{W-x}{W}}

\subsection{Equilibrium: constant fermi level}
\cgraphic{0.5}{img/pnfermilevel.png}
In equilibrium, the Fermi level mus be constant to balance transfer rates so that no net current flows.
\ceqbox{E_{F1} = E_{F2}}

\subsection{Doping Modulation}
Near the transition region, the distance between the Fermi level and conducition/valence band edge changes. Far away from the junction, the material does not ``know'' there is a junction.

$E_C$ defines the potential energy of $e^-$, and $E_V$ corresponds to the potential energy of holes (with hole kinetic energy increasing downward). By definition, force is the negative gradient of potential energy:

\ceqbox{F_{n,p}=-\Abl{E_{pot,\{n,p\}}}{x}=-\Abl{\left(\pm E_{C,V}\right)}{x} = \mp q\tilde{E}}

\subsection{Quasi Fermi Level}
Under Bias (e.g illumination), the equilibrium Fermi Level splits into two distinct \textit{Quasi Fermi Levels} caused by a slow recombination rate.
\cgraphic{0.8}{img/quasi_fermi.png}

\begin{align*}
n(x) &= N_C e^{(E_C(x)-E_{Fn})/kT}\\
p(x) &= N_V e^{(E_{Fp}-E_V(x))/kT}\\
(np)(x) &= n_i^2 e^{(E_{Fp}-E_{Fn})/kT}\\
&= N_CN_V e^{(E_{Fp}-E_{Fn})/kT} \\
&= n_i^2 e^{qV_F/kT}
\end{align*}

\end{multicols*}
\setcounter{secnumdepth}{2}
\end{document}
